\section*{Intel Homomorphic Encryption Acceleration Library for F\-P\-G\-As \par
(Intel H\-E\-X\-L for F\-P\-G\-A)}

Intel\-:registered\-: H\-E\-X\-L for F\-P\-G\-A is an open-\/source library that provides an implementation of homomorphic encryption primitives on Intel F\-P\-G\-As. Intel H\-E\-X\-L for F\-P\-G\-A targets integer arithmetic with word-\/sized primes, typically 40-\/60 bits. Intel H\-E\-X\-L for F\-P\-G\-A provides an A\-P\-I for 64-\/bit unsigned integers and targets Intel F\-P\-G\-As.

\subsection*{Contents}


\begin{DoxyItemize}
\item \href{#intel-homomorphic-encryption-acceleration-library-for-fpgas-intel-hexl-for-fpga}{\tt Intel Homomorphic Encryption Acceleration Library for F\-P\-G\-As (Intel H\-E\-X\-L for F\-P\-G\-A)}
\begin{DoxyItemize}
\item \href{#contents}{\tt Contents}
\item \href{#introduction}{\tt Introduction}
\item \href{#setting-up-environment}{\tt Setting up Environment}
\item \href{#building-intel-hexl-for-fpga}{\tt Building Intel H\-E\-X\-L for F\-P\-G\-A}
\begin{DoxyItemize}
\item \href{#dependencies}{\tt Dependencies}
\item \href{#create-build-directory-and-configure-cmake-build}{\tt Create Build Directory and Configure cmake Build}
\begin{DoxyItemize}
\item \href{#configuration-options}{\tt Configuration Options}
\end{DoxyItemize}
\item \href{#compiling-intel-hexl-for-fpga}{\tt Compiling Intel H\-E\-X\-L for F\-P\-G\-A}
\begin{DoxyItemize}
\item \href{#compiling-device-kernels}{\tt Compiling Device Kernels}
\begin{DoxyItemize}
\item \href{#compile-kernels-for-emulation}{\tt Compile Kernels for Emulation}
\item \href{#compile-kernels-for-generating-fpga-bitstream}{\tt Compile Kernels for Generating F\-P\-G\-A Bitstream}
\end{DoxyItemize}
\item \href{#compiling-host-side}{\tt Compiling Host Side}
\end{DoxyItemize}
\end{DoxyItemize}
\item \href{#installing-intel-hexl-for-fpga}{\tt Installing Intel H\-E\-X\-L for F\-P\-G\-A}
\item \href{#testing-intel-hexl-for-fpga}{\tt Testing Intel H\-E\-X\-L for F\-P\-G\-A}
\begin{DoxyItemize}
\item \href{#run-tests-in-emulation-mode}{\tt Run Tests in Emulation Mode}
\item \href{#run-tests-on-fpga-card}{\tt Run Tests on F\-P\-G\-A Card}
\end{DoxyItemize}
\item \href{#benchmarking-intel-hexl-for-fpga}{\tt Benchmarking Intel H\-E\-X\-L for F\-P\-G\-A}
\begin{DoxyItemize}
\item \href{#run-benchmarks-in-emulation-mode}{\tt Run Benchmarks in Emulation Mode}
\item \href{#run-benchmarks-on-fpga-card}{\tt Run Benchmarks on F\-P\-G\-A Card}
\end{DoxyItemize}
\item \href{#using-intel-hexl-for-fpga}{\tt Using Intel H\-E\-X\-L for F\-P\-G\-A}
\item \href{#debugging}{\tt Debugging}
\end{DoxyItemize}
\item \href{#documentation}{\tt Documentation}
\begin{DoxyItemize}
\item \href{#doxygen}{\tt Doxygen}
\end{DoxyItemize}
\item \href{#contributing}{\tt Contributing}
\begin{DoxyItemize}
\item \href{#pull-request-acceptance-criteria-pending-performance-validation}{\tt Pull request acceptance criteria (Pending performance validation)}
\item \href{#repository-layout}{\tt Repository layout}
\end{DoxyItemize}
\item \href{#citing-intel-hexl-for-fpga}{\tt Citing Intel H\-E\-X\-L for F\-P\-G\-A}
\begin{DoxyItemize}
\item \href{#version-10}{\tt Version 1.\-0}
\end{DoxyItemize}
\item \href{#contributors}{\tt Contributors}
\item \href{#contact-us}{\tt Contact us}
\end{DoxyItemize}

\subsection*{Introduction}

Many cryptographic applications, particularly homomorphic encryption (H\-E), rely on integer polynomial arithmetic in a finite field. H\-E, which enables computation on encrypted data, typically uses polynomials with degree {\ttfamily N\-:} a power of two roughly in the range {\ttfamily N=\mbox{[}2$^\wedge$\{10\}, 2$^\wedge$\{14\}\mbox{]}}. The coefficients of these polynomials are in a finite field with a word-\/sized primes, {\ttfamily p}, up to {\ttfamily p}$\sim$62 bits. More precisely, the polynomials live in the ring {\ttfamily Z\-\_\-p\mbox{[}X\mbox{]}/(X$^\wedge$\-N + 1)}. That is, when adding or multiplying two polynomials, each coefficient of the result is reduced by the prime modulus {\ttfamily p}. When multiplying two polynomials, the resulting polynomials of degree {\ttfamily 2\-N} is additionally reduced by taking the remainder when dividing by {\ttfamily X$^\wedge$\-N+1}.

The primary bottleneck in many H\-E applications is polynomial-\/polynomial multiplication in {\ttfamily Z\-\_\-p\mbox{[}X\mbox{]}/(X$^\wedge$\-N + 1)}. Intel H\-E\-X\-L for F\-P\-G\-A provides the basic primitives that allow polynomial multiplication. For efficient implementation, Intel H\-E\-X\-L for F\-P\-G\-A uses the negacyclic number-\/theoretic transform (N\-T\-T). To multiply two polynomials, {\ttfamily p\-\_\-1(x), p\-\_\-2(x)} using the N\-T\-T, we perform the forward number-\/theoretic transform on the two input polynomials, then perform an element-\/wise modular multiplication, and perform the inverse number-\/theoretic transform on the result.

Intel H\-E\-X\-L for F\-P\-G\-A implements the following functions\-:
\begin{DoxyItemize}
\item The forward and inverse negacyclic number-\/theoretic transform (N\-T\-T).
\item Dyadic multiplication.
\end{DoxyItemize}

For each function, the library provides an F\-P\-G\-A implementation using Open\-C\-L.

For additional functionality, see the public headers, located in {\ttfamily include/hexl-\/fpga.\-h}

Note\-: we provide an integrated kernel implementing the N\-T\-T/\-I\-N\-T\-T and the dyadic multiplication in one file. We also provide for convenience kernels implementing only one function stand alone. Those F\-P\-G\-A kernels work independently of each other, i.\-e. one does not require the use of another. The stand alone kernels allow testing and experimentation on a single primitive.

\subsection*{Setting up Environment}

To use this code, a prerequisite is to install a P\-C\-Ie card Intel P\-A\-C D5005 and its software stack, named Intel Acceleration Stack, which includes Quartus Prime, Intel F\-P\-G\-A S\-D\-K and Intel P\-A\-C D5005 board software package. See \href{PREREQUISITE.md}{\tt P\-R\-E\-R\-E\-Q\-U\-I\-S\-I\-T\-E.\-md} for details. If you have already installed the P\-C\-Ie card and above mentioned softwares you can skip the procedure in the link given below. \par


You can find installation instructions for the F\-P\-G\-A P\-A\-C D5005 board software package following this link\-: \par
 \href{https://www.intel.com/content/www/us/en/programmable/documentation/edj1542148561811.html}{\tt Hardware/\-Software Installation link}

Check that your installation is functional with the software environment by running the Hello F\-P\-G\-A test code as indicated in the above link. \par


\subsection*{Building Intel H\-E\-X\-L for F\-P\-G\-A}

Building Intel H\-E\-X\-L for F\-P\-G\-A library requires building all the depedencies ( mostly dealt automatically by cmake scripts) and two other separate pieces\-:
\begin{DoxyItemize}
\item Host application and related dependencies.
\item F\-P\-G\-A kernels and H\-L\-S libraries needed by the kernels.
\end{DoxyItemize}

From user point of view it is required to go through these two main steps. Without building kernels, tests, benchmark and examples cannot be launched.

\subsubsection*{Dependencies}

We have tested Intel H\-E\-X\-L for F\-P\-G\-A on the following operating systems\-: \par

\begin{DoxyItemize}
\item Centos 7.\-9.\-2009 \par

\item To check your Centos 7 version\-: \par
 ``` cat /etc/centos-\/release ```
\end{DoxyItemize}

Intel H\-E\-X\-L for F\-P\-G\-A requires the following dependencies\-:

\begin{TabularC}{2}
\hline
\rowcolor{lightgray}{\bf Dependency }&{\bf Version  }\\\cline{1-2}
Centos 7 &7.\-9.\-2009 \\\cline{1-2}
C\-Make &$>$= 3.\-5 \\\cline{1-2}
Compiler &g++ $>$= 9.\-1 \\\cline{1-2}
Doxygen &1.\-8.\-5 \\\cline{1-2}
Hardware &P\-C\-Ie Card P\-A\-C D5005 \\\cline{1-2}
\end{TabularC}
\subsubsection*{Create Build Directory and Configure cmake Build}

After cloning the git repository into your local area, you can use the following commands to set the install path and create a build directory. It will also create cmake cache files and make files that will be used for building host and kernels. Most of the build options described in previous section can be enabled or disabled by modifying the command given below\-:

``` cmake -\/\-S . -\/\-B build -\/\-D\-C\-M\-A\-K\-E\-\_\-\-I\-N\-S\-T\-A\-L\-L\-\_\-\-P\-R\-E\-F\-I\-X=./hexl-\/fpga-\/install -\/\-D\-C\-M\-A\-K\-E\-\_\-\-B\-U\-I\-L\-D\-\_\-\-T\-Y\-P\-E=Release -\/\-D\-C\-M\-A\-K\-E\-\_\-\-C\-X\-X\-\_\-\-C\-O\-M\-P\-I\-L\-E\-R=g++ -\/\-D\-C\-M\-A\-K\-E\-\_\-\-C\-\_\-\-C\-O\-M\-P\-I\-L\-E\-R=gcc -\/\-D\-E\-N\-A\-B\-L\-E\-\_\-\-F\-P\-G\-A\-\_\-\-D\-E\-B\-U\-G=O\-N -\/\-D\-E\-N\-A\-B\-L\-E\-\_\-\-T\-E\-S\-T\-S=O\-N -\/\-D\-E\-N\-A\-B\-L\-E\-\_\-\-D\-O\-C\-S=O\-N -\/\-D\-E\-N\-A\-B\-L\-E\-\_\-\-B\-E\-N\-C\-H\-M\-A\-R\-K=O\-N ```

Different cmake options are provided allowing users to configure the overall build process. With these options the user can control if it is required to build tests, benchmark etc. Note that by default all options are off\-: the user must enable at least a few options to create a useful code. The recommended options can be found below. The details of these options is given in next section with default selection\-: \par


\paragraph*{Configuration Options}

In addition to the standard C\-Make configuration options, Intel H\-E\-X\-L for F\-P\-G\-A supports several cmake options to configure the build. For convenience, they are listed below\-:

\begin{TabularC}{3}
\hline
\rowcolor{lightgray}{\bf C\-Make option }&{\bf Values }&{\bf }\\\cline{1-3}
E\-N\-A\-B\-L\-E\-\_\-\-B\-E\-N\-C\-H\-M\-A\-R\-K &O\-N / O\-F\-F (default O\-F\-F) &Set to O\-F\-F, enable benchmark suite via Google benchmark \\\cline{1-3}
E\-N\-A\-B\-L\-E\-\_\-\-F\-P\-G\-A\-\_\-\-D\-E\-B\-U\-G &O\-N / O\-F\-F (default O\-F\-F) &Set to O\-F\-F, enable debug log at large runtime penalty \\\cline{1-3}
E\-N\-A\-B\-L\-E\-\_\-\-T\-E\-S\-T\-S &O\-N / O\-F\-F (default O\-F\-F) &Set to O\-F\-F, enable building of unit-\/tests \\\cline{1-3}
E\-N\-A\-B\-L\-E\-\_\-\-D\-O\-C\-S &O\-N / O\-F\-F (default O\-F\-F) &Set to O\-F\-F, enable building of documentation \\\cline{1-3}
\end{TabularC}
\subsubsection*{Compiling Intel H\-E\-X\-L for F\-P\-G\-A}

Compiling H\-E\-X\-L for F\-P\-G\-A requires two steps\-: compiling the C++ host code and compiling the Open\-C\-L kernels. Start by compiling the kernels as they will be needed during the host installation. Before proceeding to the compilations and installation, make sure that your environment variables are set according to the instructions in the Intel P\-A\-C\-D5005 Software Package installation guide.

\paragraph*{Compiling Device Kernels}

The kernels can be compiled in two different modes, emulation and F\-P\-G\-A. The emulation mode runs the kernels on the C\-P\-U. Compiling in emulation mode takes only a few minutes. The resulting bitstream can be used to verify the functionality of kernels on the C\-P\-U. The F\-P\-G\-A mode builds the kernel bitstream for F\-G\-P\-A card. Compiling the kernels in F\-P\-G\-A mode can take a few hours.

\subparagraph*{Compile Kernels for Emulation}

To compile the device kernel for running in emulation mode\-: \par
 ``` cmake --build build --target emulation ```

This command takes a few minutes to execute.

\begin{quotation}
$\ast$$\ast$\-\_\-\-N\-O\-T\-E\-:\-\_\-$\ast$$\ast$ If you are interested to run kernels in software emulation mode only then this step is enough and you can move to building the host code. If you want to run the kernels on actual F\-P\-G\-A board please follow the next steps for building bitstream for the F\-P\-G\-A card.

\end{quotation}


\subparagraph*{Compile Kernels for Generating F\-P\-G\-A Bitstream}

To compile the device kernel in fpga mode\-: \par
 ``` cmake --build build --target fpga ``` This command takes a few hours to execute.

The bitstreams will be located in the installation directory specified when calling the cmake command.(See installation below) \par


\paragraph*{Compiling Host Side}

To build the host application, tests, benchmark, and documentation (depending on the options selected above) run the following command\-: ``` cmake --build build ```

This will build the Intel H\-E\-X\-L for F\-P\-G\-A library in the {\ttfamily build/host/} directory. \par


\subsection*{Installing Intel H\-E\-X\-L for F\-P\-G\-A}

After compiling both host side and device kernels, users need to install H\-E\-X\-L for F\-P\-G\-A as a standalone library. The library is used for building and running H\-E\-X\-L for F\-P\-G\-A tests and benchmarks, and it can also be used as a third-\/party library. To install Intel H\-E\-X\-L for F\-P\-G\-A to the installation directory specified at configuration time\-: \par
 ``` cmake --install build ```

\subsection*{Testing Intel H\-E\-X\-L for F\-P\-G\-A}

To run a set of unit tests via Googletest run the following command ( for running the test you should have chosen {\ttfamily -\/\-D\-E\-N\-A\-B\-L\-E\-\_\-\-T\-E\-S\-T\-S=O\-N} otherwise tests may not be enabled) (see \href{#configuration-options}{\tt Configuration Options}). \par
 Make sure that the .aocx files have been installed in the install directory that was chosen during configuration. The default choice we made was \char`\"{}./hexl-\/fpga-\/install\char`\"{}. \par


\subsubsection*{Run Tests in Emulation Mode}

In emulation mode the kernel will run on the C\-P\-U and the user will be able to test and validate the kernel. \par
 To run in emulation mode (setting R\-U\-N\-\_\-\-C\-H\-O\-I\-C\-E to different values informs host code about emulation mode or F\-P\-G\-A run)\-: \par


``` export R\-U\-N\-\_\-\-C\-H\-O\-I\-C\-E=1 cmake --build build --target tests ```

\subsubsection*{Run Tests on F\-P\-G\-A Card}

To run using actual F\-P\-G\-A card, run the following command (setting R\-U\-N\-\_\-\-C\-H\-O\-I\-C\-E to different values informs host code about emulation mode or F\-P\-G\-A run)\-: \par


``` export R\-U\-N\-\_\-\-C\-H\-O\-I\-C\-E=2 cmake --build build --target tests ``{\ttfamily  The tests executables are located in}build/tests/` directory \par


\subsection*{Benchmarking Intel H\-E\-X\-L for F\-P\-G\-A}

To run a set of benchmarks via Google benchmark, configure and build Intel H\-E\-X\-L for F\-P\-G\-A with {\ttfamily -\/\-D\-E\-N\-A\-B\-L\-E\-\_\-\-B\-E\-N\-C\-H\-M\-A\-R\-K=O\-N} (see \href{#configuration-options}{\tt Configuration Options}). \par
 Make sure that the .aocx files have been installed in {\ttfamily $<$chosen install directory$>$/bench/} directory. \par


\subsubsection*{Run Benchmarks in Emulation Mode}

To run the benchmark in emulation mode\-: \par
 ``` export R\-U\-N\-\_\-\-C\-H\-O\-I\-C\-E=1 cmake --build build --target bench ``` \subsubsection*{Run Benchmarks on F\-P\-G\-A Card}

To run the benchmark on the fpga, run \par
 ``` export R\-U\-N\-\_\-\-C\-H\-O\-I\-C\-E=2 cmake --build build --target bench ``{\ttfamily  The benchmark executables are located in}build/benchmark/` directory \par


\subsection*{Using Intel H\-E\-X\-L for F\-P\-G\-A}

The {\ttfamily examples} folder contains an example showing how to use Intel H\-E\-X\-L for F\-P\-G\-A library in a third-\/party project. See \href{examples/README.md}{\tt examples/\-R\-E\-A\-D\-M\-E.\-md} for details. \par


\subsection*{Debugging}

For optimal performance, Intel H\-E\-X\-L for F\-P\-G\-A does not perform input validation. In many cases the time required for the validation would be longer than the execution of the function itself. To debug Intel H\-E\-X\-L for F\-P\-G\-A, configure and build Intel H\-E\-X\-L for F\-P\-G\-A with the option \par
 {\ttfamily -\/\-D\-C\-M\-A\-K\-E\-\_\-\-B\-U\-I\-L\-D\-\_\-\-T\-Y\-P\-E=Debug} This will generate a debug version of the library that can be used to debug the execution. To enable the F\-P\-G\-A logs, configure the build with {\ttfamily -\/\-D\-E\-N\-A\-B\-L\-E\-\_\-\-F\-P\-G\-A\-\_\-\-D\-E\-B\-U\-G=O\-N} (see \href{#configuration-options}{\tt Configuration Options}). \par


\begin{quotation}
$\ast$$\ast$\-\_\-\-N\-O\-T\-E\-:\-\_\-$\ast$$\ast$ Enabling {\ttfamily -\/\-D\-C\-M\-A\-K\-E\-\_\-\-B\-U\-I\-L\-D\-\_\-\-T\-Y\-P\-E=Debug} will result in a significant runtime overhead. \par


\end{quotation}


\section*{Documentation}

See \href{https://intel.github.io/hexl-fpga}{\tt https\-://intel.\-github.\-io/hexl-\/fpga} for Doxygen documentation. \par


Intel H\-E\-X\-L for F\-P\-G\-A supports documentation via Doxygen. To build documentation, first install {\ttfamily doxygen} and {\ttfamily graphviz}, e.\-g. ```bash sudo yum install doxygen graphviz ``{\ttfamily  Then, configure Intel H\-E\-X\-L for F\-P\-G\-A with}-\/\-D\-E\-N\-A\-B\-L\-E\-\_\-\-D\-O\-C\-S=O\-N` (see \href{#configuration-options}{\tt Configuration Options}). \subsubsection*{Doxygen}

To build Doxygen documentation, after configuring Intel H\-E\-X\-L for F\-P\-G\-A with {\ttfamily -\/\-D\-E\-N\-A\-B\-L\-E\-\_\-\-D\-O\-C\-S=O\-N}, run ``` cmake --build build --target docs ``{\ttfamily  To view the generated Doxygen documentation, open the generated}build/doc/doxygen/html/index.\-html` file in a web browser. \begin{quotation}
$\ast$$\ast$\-\_\-\-N\-O\-T\-E\-:\-\_\-$\ast$$\ast$ After running the cmake --install build command, the documentation will also be available in\-: \par


\end{quotation}
{\ttfamily $<$chosen install directory$>$/doc/doxygen/html/index.html}.

\section*{Contributing}

At this time, Intel H\-E\-X\-L for F\-P\-G\-A welcomes external contributions. To contribute to Intel H\-E\-X\-L for F\-P\-G\-A, see \href{CONTRIBUTING.md}{\tt C\-O\-N\-T\-R\-I\-B\-U\-T\-I\-N\-G.\-md}. We encourage feedback and suggestions via Github Issues as well as discussion via Github Discussions.

Please use \href{https://pre-commit.com/}{\tt pre-\/commit} to validate the formatting of the code before submitting a pull request. \par
 To install pre-\/commit\-: \par
 ``` pip install --user cpplint pre-\/commit ``` To run pre-\/commit\-: ``` pre-\/commit run --all ```

\subsection*{Pull request acceptance criteria (Pending performance validation)}

Pull requests will be accepted if they provide better acceleration, fix a bug or add a desirable new feature.

Before contributing, please run ```bash cmake --build build --target tests ``` and make sure pre-\/commit checks and all unit tests pass. \par


``` pre-\/commit run --all ```

\subsection*{Repository layout}

Public headers reside in the {\ttfamily hexl-\/fpga-\/install/include} folder. Private headers, e.\-g. those containing fpga code should not be put in this folder.

\section*{Citing Intel H\-E\-X\-L for F\-P\-G\-A}

To cite Intel H\-E\-X\-L for F\-P\-G\-A, please use the following Bib\-Te\-X entry.

\subsubsection*{Version 1.\-0}

```tex \{Intel\-H\-E\-X\-L\-F\-P\-G\-A, author=\{Meng,Yan and de Souza, Fillipe and Butt, Shahzad and de Lassus, Hubert and Gonzales Aragon, Tomas and Zhou, Yongfa and Wang, Yong and others\}, title = \{\{I\}ntel \{Homomorphic Encryption Acceleration Library for F\-P\-G\-As for F\-P\-G\-A\} (release 1.\-0)\}, howpublished = \{\{\href{https://github.com/intel/hexl-fpga}{\tt https\-://github.\-com/intel/hexl-\/fpga}\}\}, month = December, year = 2021, key = \{Intel H\-E\-X\-L for F\-P\-G\-A\} \} ```

\section*{Contributors}

The Intel contributors to this project, sorted by last name, are
\begin{DoxyItemize}
\item \href{https://www.linkedin.com/in/paky-abu-alam-89797710/}{\tt Paky Abu-\/\-Alam}
\item \href{https://www.linkedin.com/in/tomas-gonzalez-aragon/}{\tt Tomas Gonzalez Aragon}
\item \href{https://www.linkedin.com/in/flavio-bergamaschi-1634141/}{\tt Flavio Bergamaschi}
\item \href{https://www.linkedin.com/in/shahzad-ahmad-butt-4b44971b/}{\tt Shahzad Butt}
\item \href{https://www.linkedin.com/in/hubert-de-lassus/}{\tt Hubert de Lassus}
\item \href{https://www.linkedin.com/in/fillipe-d-m-de-souza-a8281820/}{\tt Fillipe D. M. de Souza}
\item \href{https://www.linkedin.com/in/anil-goteti}{\tt Anil Goteli}
\item \href{https://www.linkedin.com/in/jingyi-jin-655735/}{\tt Jingyi Jin}
\item \href{https://www.linkedin.com/in/yan-meng-5832895/}{\tt Yan Meng}
\item \href{https://www.linkedin.com/in/nir-peled-4a52266/}{\tt Nir Peled}
\item \href{https://github.com/wangyon1/}{\tt Yong Wang}
\item \href{https://www.linkedin.com/in/yongfa-zhou-16217166/}{\tt Yongfa Zhou}
\end{DoxyItemize}

\section*{Contact us}


\begin{DoxyItemize}
\item \href{mailto:he_fpga_support@intel.com}{\tt he\-\_\-fpga\-\_\-support@intel.\-com} 
\end{DoxyItemize}